% Ao menos uma linguagem (brazil ou english) deveria sempre ser fornecida
\documentclass[brazil,nolapesd,aspectratio=169,noartschool]{lapesd-slides}

\usepackage[stable]{footmisc}
\usepackage[utf8]{inputenc}
\usepackage[T1]{fontenc}
\usepackage{hyperref}
\usepackage[absolute,overlay]{textpos}
\usepackage{todonotes}
\usepackage{babel}
\usepackage[acronym]{glossaries}
\usepackage{abntex2cite}
\usepackage[useregional]{datetime2}

\newcommand{\mydate}{\DTMdisplaydate{2020}{10}{28}{-1}}

%%%%%%%%%%%%%%%%%%%%%%%%%%%%
% Metadados
%%%%%%%%%%%%%%%%%%%%%%%%%%%%

\title[Risc-V I/O]{Seminar H - I/O Operation}
\author[Silva and Uller]{\large Guilherme Antonio Ferreira da Silva\\João Fellipe Uller}

\institute{
  \fontsize{10.5}{12.6}\selectfont 
  Depto. de Informática e Estatística\\
  Universidade Federal de Santa Catarina - Florianópolis\\
  INE5424 - Sistemas Operacionais II\\
}

\date{\mydate}

%%%%%%%%%%%%%%%%%%%%%%%%%%%%
% Slides
%%%%%%%%%%%%%%%%%%%%%%%%%%%%

\begin{document}

\titleframe

% Você não é obrigado a colocar um sumário!
\begin{frame}{Sumário}
  \tableofcontents
\end{frame}

% Desse ponto em diante serão inseridos slides de pausa a cada \section
\hidesections

\section{Introdução}


\section{Opções da classe}


\section{Novos comandos e Ambientes}


\section{Exemplos}

\begin{frame}{Figuras}
  \addfiglw[Art school building (Chair, 2019)]{\jobname-ufsc.jpg}
  
  \begin{itemize}
    \item Figura inclusa com \mt|\addfiglw|
  \end{itemize}
\end{frame}


\begin{frame}{VirtIO}
É um protocolo de comunicação entre dispositivos virtuais, focado em três objetivos:

\begin{itemize}
  \item Definir uma API para a comunição de dispositivos
  \item Unificar drivers
  \item Facilitar o uso e configuração de dispositivos em todos os sistemas de virtualização existentes
\end{itemize}


Para utilizar esse protocolo, precisamos de três componentes:
\begin{itemize}
  \item Device status field
  \item Feature field
  \item Virtqueue
\end{itemize}
\end{frame}

\begin{frame}{Device status field}
É uma sequência de bits utilizada pelo dispositivo em conjunto com o sistema operacional para iniciar a comunicação entre estes.


\begin{itemize}
  \item ACKNOWLEDGE - Dispositivo reconhecido.
  \item DRIVER - Inicialização.
  \item DRIVER{\_}OK - Sinaliza que a comunicação está pronta para começar.
  \item FEATURES{\_}OK - Acordo sobre quais features serão usadas finalizado.
  \item DEVICE{\_}NEEDS{\_}RESET - Falha fatal do lado do dispositivo.
  \item FAILED - Falha fatal do lado do sistema operacional.
\end{itemize}
\end{frame}

\begin{frame}{Feature field}
Já o feature field é utilizado entre o sistema operacional e o dispositivo para saber quais features estão disponíveis e quais serão utilizadas.

Para realizar essa negociação, são utilizados dois registradores:
\begin{itemize}
  \item guest{\_}feature
  \item host{\_}feature
\end{itemize}


Quando essa negociação é finalizada, usamos o device status field e setamos o bit de FEATURES{\_}OK.
\end{frame}

\begin{frame}{Virtqueue}
A interface alocada na memória do dispositivo que faz o transporte entre dois grupos:

\begin{itemize}
  \item Front-end - Realizadas as requisições e enviadas para o back-end.
  \item Back-end - Recebe as requisições e executa-as
\end{itemize}

% TODO: Adicionar imagem da arquiteura aqui.

\end{frame}

\begin{frame}{Vring}
Estrutura que implementa de fato a Virtqueue, divididade em três componentes:

\begin{itemize}
  \item Array de descriptors
  \item Available Ring
  \item Used Ring
\end{itemize}
\end{frame}

\begin{frame}{Descriptors}
  Aqui são inseridos os dados que serão lidos/escritos eventualmente.

  São formados por campo contendo seu tamanho e um endereço de 64 bits, um campo next que aponta
  para o próximo descriptor, se existir e um campo de flag.

  \begin{itemize}
    \item VIRTQ{\_}DESC{\_}F{\_}WRITE - Indica se está no modo leitura ou escrita.
    \item VIRTQ{\_}DESC{\_}F{\_}NEXT - Indica se existe um encadeamento de buffers.
  \end{itemize}
\end{frame}

\begin{frame}{Available ring e Used Ring}
  Após a preencher o descriptor, colocamos ele dentro de um Available ring. Ele e o used ring são utilizados
  para trocar as informações entre sistema operacional e dispositivo.Ambos possuem um campo de flag para desabilitar interrupções, além de seu indice.


  Tanto o sistema operacional quanto o dispositivo acompanham o estado desses dois anéis, para garantir que estejam falando sobre a mesma requisição.
\end{frame}

\begin{frame}{Ciclo de uma requisição}
  Inicialmente, precisamos iniciar a comunicação e verificar o tipo de dispositivo, já que o VirtIO define protocolos específicos para cada tipo de dispositivo.


  \begin{enumerate}
    \item Setar os bits necessários dentro do device status field.
    \item Setar as features a serem utilizadas via feature field.
    \item Varrer o barramento procurando pelo Magic Value, i.e, 'virt'
    \item Ler o VendorID e identificar o tipo de dispositivo.
  \end{enumerate}
\end{frame}

\begin{frame}{Ciclo de uma requisição}
Encontrado o tipo de dispositivo, precisamos preencher os descriptors de acordo com seu tipo. No caso de um block device, por exemplo,
precisariamos de três descriptors: um para o header, um para o buffer e um para o status.

\begin{itemize}
  \item header - Responsável por indicar se dispositivo vai ler ou escrever
  \item buffer - Valor que será escrito na memória ou então endereço de memória a ser lido.
  \item status - Guarda o resultado da requisição, que pode ser 0 - success, 1 - failure ou 2 - unsupported device.
\end{itemize}
\end{frame}

\begin{frame}{Ciclo de uma requisição}
Preenchido os descriptors, escrevemos o valor 0 no registrador queue{\_}notify para avisar que a requisição deve ser iniciada,
inserindo também os descriptors no available ring.


Ao finalizar a execução, o dispositivo então emite uma interrupção e recebemos de volta um used ring, contendo o id do descriptor usado,
já que as operações podem terminar em ordem diferente do que foram enviadas. Por fim, validado o descriptor e recebida a resposta, liberamos a memória alocada.
\end{frame}

%%%%%%%%%%%%%%%%%%%%%%%%%%%%
% Finalização
%%%%%%%%%%%%%%%%%%%%%%%%%%%%

\thanksframe

\referencesframe{userguide}

\end{document}
