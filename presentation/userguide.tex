% Ao menos uma linguagem (brazil ou english) deveria sempre ser fornecida
\documentclass[brazil,nolapesd,aspectratio=169,noartschool]{lapesd-slides}

\usepackage[stable]{footmisc}
\usepackage[utf8]{inputenc}
\usepackage[T1]{fontenc}
\usepackage{hyperref}
\usepackage[absolute,overlay]{textpos}
\usepackage{todonotes}
\usepackage{babel}
\usepackage[acronym]{glossaries}
\usepackage{abntex2cite}
\usepackage[useregional]{datetime2}

\graphicspath{{../img/}}
\DeclareGraphicsExtensions{.pdf,.jpeg,.png}

\newcommand{\mydate}{\DTMdisplaydate{2020}{10}{28}{-1}}

%%%%%%%%%%%%%%%%%%%%%%%%%%%%
% Metadados
%%%%%%%%%%%%%%%%%%%%%%%%%%%%

\title[Risc-V I/O]{Seminar H - I/O Operation}
\author[Silva and Uller]{\large Guilherme Antonio Ferreira da Silva\\João Fellipe Uller}

\institute{
  \fontsize{10.5}{12.6}\selectfont 
  Depto. de Informática e Estatística\\
  Universidade Federal de Santa Catarina - Florianópolis\\
  INE5424 - Sistemas Operacionais II\\
}

\date{\mydate}

%%%%%%%%%%%%%%%%%%%%%%%%%%%%
% Slides
%%%%%%%%%%%%%%%%%%%%%%%%%%%%

\begin{document}

\titleframe

% Você não é obrigado a colocar um sumário!
\begin{frame}{Sumário}
  \tableofcontents
\end{frame}

% Desse ponto em diante serão inseridos slides de pausa a cada \section
\showsections

\section{RISC-V I/O}

	\begin{frame}{RISC-V I/O}
		\textit{Memory Mapped I/O}
		\begin{itemize}
			\item Mesmo espaço de endereçamento para endereçar memória e I/O
			\item Utiliza as instruções regulares de acesso à memória
		\end{itemize}

		\vspace{1em}

		Vantagens:
		\begin{itemize}
			\item ISA simplificada
			\item CPU menor
			\item Maior proteção
		\end{itemize}
	\end{frame}


	\subsection{MMIO}

	%
	\begin{frame}{\textit{Memory Mapped I/O}}
		Dispositivos de I/O mapeados dentro do espaço de endereçamento do sistema
		\begin{itemize}
			\item Monitoram o barramento de endereço da CPU esperando requisições
		\end{itemize}

		\addfig[][width=0.35\linewidth]{mmio_layout.png}
	\end{frame}


	\subsection{Mapeamento no QEMU}

	%
	\begin{frame}{Mapeamento no QEMU}
		\begin{itemize}
			\item Mapeamento fica a cargo do ambiente de execução
			\begin{itemize}
				\item Normalmente realizado durante o boot
			\end{itemize}

			\vspace{0.5em}
			\item Mapeamento dos registradores de \textit{hardware} dos dispositivos
				para dentro do espaço de endereçamento
		\end{itemize}
	\end{frame}

	%
	\begin{frame}{Mapeamento no QEMU}
		\begin{itemize}
			\item VirtIO
			\item Endereço 0x10001000 até 0x10008000
		\end{itemize}
		\addfig[][width=0.85\linewidth]{qemu_map.png}
	\end{frame}


	\subsection{Acesso}

	%
	\begin{frame}{Acesso às regiões de I/O}
		\begin{itemize}
			\item \textit{Uncached} load/stores
			\begin{itemize}
				\item Dados buscados diretamente dos controladores dos dispositivos
			\end{itemize}

			\vspace{0.5em}
			\item Regiões inacessíveis diretamente para \textit{threads} do espaço de usuário
		\end{itemize}
	\end{frame}


	\subsection{Ordenamento}

	%
	\begin{frame}{Ordenamento de operações}
		Modelo de memória relaxado
		\begin{itemize}
			\item Sem garantia de ordem entre threads
		\end{itemize}

		\vspace{1em}

		Soluções para obter ordenamento rígido:
		\begin{itemize}
			\item \textit{Channels}
			\item FENCE
		\end{itemize}
	\end{frame}

	%
	\begin{frame}{Channels}
		\begin{itemize}
			\item Mecanismo que possibilita ordenamento entre diferentes regiões
			\item Identificador numérico
			\begin{itemize}
				\item Canal 0: ordenamento ponto-a-ponto entre a \textit{thread} e a região específicada
				\item Canal 1: ordenamento global entre todas as regiões de I/O
			\end{itemize}
		\end{itemize}
	\end{frame}

	%
	\begin{frame}{FENCE}
		\addfig[][width=\linewidth]{fence_spec.png}

		Garante que as operações do grupo \textit{predecessor} serão vistas antes de qualquer operação
		do grupo \textit{sucessor} à instrução FENCE

		\vspace{1em}

		\begin{itemize}
			\item Ordenamento entre qualquer combinação de operações:
			\begin{itemize}
				\item Acessos à memória: R e W
				\item Entrada e saída: I e O
			\end{itemize}
		\end{itemize}
	\end{frame}


\section{VirtIO}

\begin{frame}{VirtIO}
	Protocolo de comunicação entre dispositivos virtuais, focado em três objetivos:

	\begin{itemize}
		\item Definir uma API para a comunição de dispositivos
		\item Unificar drivers
		\item Facilitar o uso e configuração de dispositivos em todos os sistemas de virtualização existentes
	\end{itemize}

	Três componentes básicos:
	\begin{itemize}
		\item Device status field
		\item Feature field
		\item Virtqueue
	\end{itemize}
\end{frame}

\begin{frame}{Device status field}
	Sequência de bits utilizada pelo dispositivo em conjunto com o SO para iniciar a comunicação entre estes.

	\begin{itemize}
		\item ACKNOWLEDGE - Dispositivo reconhecido.
		\item DRIVER - Inicialização.
		\item DRIVER{\_}OK - Comunicação pronta para começar.
		\item FEATURES{\_}OK - Acordo sobre as features finalizado.
		\item DEVICE{\_}NEEDS{\_}RESET - Falha fatal do lado do dispositivo.
		\item FAILED - Falha fatal do lado do SO.
	\end{itemize}
\end{frame}

\begin{frame}{Feature field}
	Utilizado entre o SO e o dispositivo para saber quais features estão disponíveis e quais serão utilizadas.

	Para essa negociação são utilizados dois registradores:
	\begin{itemize}
		\item guest{\_}feature
		\item host{\_}feature
	\end{itemize}

	Quando a negociação é finalizada, usamos o device status field e setamos o bit de FEATURES{\_}OK.
\end{frame}

\begin{frame}{Virtqueue}
	Interface na memória do dispositivo que faz o transporte entre dois grupos:

	\begin{itemize}
		\item Front-end - Realiza as requisições e as envia para o back-end.
		\item Back-end - Recebe as requisições e as executa
	\end{itemize}

	\addfig[][width=0.65\linewidth]{virtio-arch.png}

\end{frame}

\begin{frame}{Vring}
	Estrutura que implementa de fato a Virtqueue, dividida em três componentes:

	\begin{itemize}
		\item Array de descriptors
		\item Available Ring
		\item Used Ring
	\end{itemize}
\end{frame}

\begin{frame}{Descriptors}
	Local para inserção dos dados que serão lidos/escritos eventualmente.

	\vspace{1em}

	Formados por um campo contendo seu tamanho e um endereço de 64 bits, um campo next que aponta
	para o próximo descriptor, se existir e um campo flag.

	\vspace{1em}

	\begin{itemize}
		\item VIRTQ{\_}DESC{\_}F{\_}WRITE - Indica se está no modo leitura ou escrita.
		\item VIRTQ{\_}DESC{\_}F{\_}NEXT - Indica se existe um encadeamento de buffers.
	\end{itemize}
\end{frame}

\begin{frame}{Available ring e Used Ring}
	Após preencher o descriptor, colocamos ele dentro de um Available ring. Ele e o used ring são
	utilizados para trocar as informações entre o SO e o dispositivo. Ambos possuem um campo de
	flag para desabilitar interrupções, além de seu indice.

	\vspace{1em}

	Tanto o SO quanto o dispositivo acompanham o estado desses dois aneis, para garantir que estejam
	tratando a mesma requisição.
\end{frame}

\begin{frame}{Ciclo de uma requisição}
	VirtIO define protocolos específicos para cada tipo de dispositivo.
	\begin{itemize}
		\item Necessidade de verificar o tipo de dispositivo na inicialização da comunicação
	\end{itemize}

	\vspace{1em}

	\begin{enumerate}
		\item Setar os bits necessários dentro do device status field.
		\item Setar as features a serem utilizadas via feature field.
		\item Varrer o barramento procurando pelo Magic Value, i.e, 'virt'
		\item Ler o VendorID e identificar o tipo de dispositivo.
	\end{enumerate}
\end{frame}

\begin{frame}{Ciclo de uma requisição}
	Descriptors devem ser preenchidos de acordo com seu tipo.
	
	\vspace{1em}

	Para um \textit{Block device}, por exemplo, precisaria-se de três descriptors:
	\begin{itemize}
		\item header - Indica se o dispositivo vai ler ou escrever
		\item buffer - Valor que será escrito na memória ou o endereço de memória a ser lido.
		\item status - Guarda o resultado da requisição, que pode ser 0 - success, 1 - failure ou 2 - unsupported device.
	\end{itemize}
\end{frame}

\begin{frame}{Ciclo de uma requisição}
	Em seguida, escrevemos o valor 0 no registrador queue{\_}notify para avisar que a requisição deve ser
	iniciada, inserindo os descriptors no available ring.

	\vspace{1em}

	Ao finalizar a execução, o dispositivo emite uma interrupção e recebemos de volta um used ring,
	contendo o id do descriptor usado, já que as operações podem terminar em ordem diferente do que
	foram enviadas. Por fim, validado o descriptor e recebida a resposta, liberamos a memória alocada.

\end{frame}

%%%%%%%%%%%%%%%%%%%%%%%%%%%%
% Finalização
%%%%%%%%%%%%%%%%%%%%%%%%%%%%

\thanksframe

\nocite{RISCVUserISA}
\nocite{virtioPaper}
\nocite{virtioManual}

\referencesframe{userguide}

\end{document}
