% Ao menos uma linguagem (brazil ou english) deveria sempre ser fornecida
\documentclass[brazil,nolapesd,aspectratio=169,noartschool]{lapesd-slides}

\usepackage[stable]{footmisc}
\usepackage[utf8]{inputenc}
\usepackage[T1]{fontenc}
\usepackage{hyperref}
\usepackage[absolute,overlay]{textpos}
\usepackage{todonotes}
\usepackage{babel}
\usepackage[acronym]{glossaries}
\usepackage{abntex2cite}
\usepackage[useregional]{datetime2}

\newcommand{\mydate}{\DTMdisplaydate{2020}{10}{28}{-1}}

%%%%%%%%%%%%%%%%%%%%%%%%%%%%
% Metadados
%%%%%%%%%%%%%%%%%%%%%%%%%%%%

\title[Risc-V I/O]{Seminar H - I/O Operation}
\author[Silva and Uller]{\large Guilherme Antonio Ferreira da Silva\\João Fellipe Uller}

\institute{
  \fontsize{10.5}{12.6}\selectfont 
  Depto. de Informática e Estatística\\
  Universidade Federal de Santa Catarina - Florianópolis\\
  INE5424 - Sistemas Operacionais II\\
}

\date{\mydate}

%%%%%%%%%%%%%%%%%%%%%%%%%%%%
% Slides
%%%%%%%%%%%%%%%%%%%%%%%%%%%%

\begin{document}

\titleframe

% Você não é obrigado a colocar um sumário!
\begin{frame}{Sumário}
  \tableofcontents
\end{frame}

% Desse ponto em diante serão inseridos slides de pausa a cada \section
\hidesections

\section{Introdução}


\section{Opções da classe}


\section{Novos comandos e Ambientes}


\section{Exemplos}

\begin{frame}{Figuras}
  \addfiglw[Art school building (Chair, 2019)]{\jobname-ufsc.jpg}
  
  \begin{itemize}
    \item Figura inclusa com \mt|\addfiglw|
  \end{itemize}
\end{frame}


%%%%%%%%%%%%%%%%%%%%%%%%%%%%
% Finalização
%%%%%%%%%%%%%%%%%%%%%%%%%%%%

\thanksframe

\referencesframe{userguide}

\end{document}
