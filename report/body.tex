%----------------------------------------------------------------------------------------
%	INTRODUCTION
%----------------------------------------------------------------------------------------

\section*{RISC-V I/O}

	% Introduction
	Para lidar com dispositivos de entrada/saída, a ISA do RISC-V não prevê nenhum tipo
	de instrução especial para controle ou leitura/escrita de um barramento. Ao invés
	disso, é previsto um esquema de entrada e saída mapeado em memória (Memory Mapped I/O).
	Assim, cada um dos dispositivos de entrada e saída é mapeado para um ou mais endereços
	no espaço de endereçamento dos processos, que passam a ser monitorados por esses
	dispositivos a fim de interceptar requisições de I/O para os mesmos.

	% Evaluation
	Desse modo, operações de entrada e saída são realizadas utilizando as instruções de
	load/store comuns que lidam com manipulação de dados da memória, permitindo a utilização
	de todos os modos de acesso previstos na especificação. Isso simplifica a programação, quando
	comparado com acessos diretos aos dispositivos, além de forçar proteção de acessos
	através do espaço de endereçamento, evitando que \textit{threads} do usuário acessem esses
	endereços diretamente. Por outro lado, esse tipo de solução incorre na necessidade
	de circuitos de decodificação em \textit{hardware} mais complexos, uma vez que precisa-se
	de uma decodificação total dos endereços virtuais para físicos para sua utilização.

	% DMA
	Além disso, a utilização de entrada e saída mapeada em memória permite a configuração e a
	utilização de uma Direct Memory Access (DMA) para realizar as operações de transferência
	de dados entre os registradores do dispositivo e o buffer de memória. Reduzindo,
	dessa forma, o tempo perdido pela CPU bloqueando à espera de uma leitura ou uma escrita na
	memória e aumentando a performance do sistema como um todo.


	\subsection*{Memory Mapped I/O (MMIO)}


	\subsection*{I/O Ordering}

%----------------------------------------------------------------------------------------